\documentclass[9pt,twocolumn,twoside]{article}
\usepackage[utf8]{inputenc}
\usepackage{graphicx}
\usepackage{amsmath}
\usepackage{hyperref}
\usepackage{natbib}
\usepackage{geometry}
\geometry{margin=1in}
\usepackage{subcaption}
\usepackage{pdfpages}
\usepackage{tikz}
\usetikzlibrary{3d, arrows.meta}
\usepackage[none]{hyphenat}

\title{Bike Design Finalised Project Plan \\
"Computer modelling of bike components to explore the viability of flax-carbon hybrid composites"
}
\author{Jack Henry
\\ crsid: jbh48}
\date{\today}

\begin{document}



\includepdf[pages=-]{BD_Plan_Project_Coversheet_Feedback_2020.pdf}

\maketitle

\setcounter{page}{1}

\section{Introduction and Aim}

Natural fibres are increasingly being adopted across various industries due to their sustainability, low weight, and good damping properties \cite{Bcomp}. However, their application in the cycling industry remains limited. 

This project aims to develop a computer model of flax-carbon hybrid composite laminates for use in bicycle components, with a focus on the front fork. The model will help predict structural and dynamic performance, serving as a benchmark for experimental validation and guiding future design and research efforts.

\section{Objectives and Deliverables}

\subsection*{Objectives}
To achieve the overall aim, the following objectives have been defined:

\begin{itemize}
  \item Conduct a literature review of natural fibre composites, especially flax-based materials, as well as other fibre reinforced polymers (FRP).
  \item Study the mechanical properties and vibration characteristics of flax (FFRP) and carbon fibre (CFRP) composites, to be used in the modelling.
  \item Learn to use composite simulation software (such as a composite compressive strength modeller, CCSM) and establish a reliable workflow for modelling composite components.
  \item Simulate various flax-carbon ratios and fibre orientations in the model and evaluate their impact on constraints such as: bending stiffness and strength, weight, vibration damping and natural frequencies, and cost.
  \item Suggest optimal material configurations for the fork design based on simulation results and constraint requirements.
\end{itemize}

\subsection*{Deliverables}
The key outputs and deliverables from the project are as follows:

\begin{itemize}
  \item A literature review summarising the current state of flax-carbon composites in structural applications.
  \item A validated simulation process using CCSM or equivalent software for evaluating hybrid composite performance.
  \item A comprehensive report of simulation results, including material comparison charts (such as carpet plots for flax-carbon composition) and design trade-offs/suggestions.
\end{itemize}

\section{Methodologies and resources}

Material properties will be gathered from literature. These material properties will then be used in a CCSM software to build various laminate arrangements. This software is used to analyse the elastic properties, deformation, and failure. Subject to time constraints, a Python model will be developed to analyse the behaviour of the model in more detail, such as vibrations. These computer models will use a simplified version of the bike components. The first component will be the bike fork, as this is a key component for strength and vibration. This will be modelled as a simple tube at first, which can again be simplified to a laminate such as in figure \ref{fig:tube}. A force can be axially applied to one of the ends to simulate the force from the handlebars/road. A transverse force should also be used to assess robustness.

If this initial simulation and analysis is successful (and in good time), there is the option of exploring other components, such as the pedal crank or seat post, or increasing complexity of the model and analysis.

The CCSM software has been preliminarily tested for a simple carbon fibre layup, acquiring information such as effective moduli for the laminate stiffnesses. If feasible, this can be used with flax fibre layers to create a carpet plot for the effect of flax-carbon arrangements on various properties (however more familiarity is required with the software to see if this is feasible).

\begin{figure}[h!]
    \centering
    \includegraphics[width=0.5\textwidth]{tube.png} 
    \caption{Tube simplification}
    \label{fig:tube}
\end{figure}

\section{Timeline}

Figure \ref{fig:Gantt} shows a Gantt chart with the proposed timeline of the main stages. It shows the key deadlines, the design stages which involve development of the model and deliverables, and writing stages which involve the parts of the project detached from the model.

\subsection{Design stages}

\subsubsection{Initial design}

Initially, a literature review will be taken place. Alongside this, material properties will be collated to be used in the model, as well as initial exposure to the model development.

Key tasks:
\begin{itemize}
    \item Collate material properties needed for modelling
    \item Define constraints to be used in the model
    \item Learn software and explore the use of applying hybrid laminates in the model
    \item Explore creation of python model for extended analysis
\end{itemize}

\subsubsection{Conceptual design}

\begin{itemize}
    \item Use material properties and treat each ply independently 
    \item Use merit indices to aid material selection for individual constraints
\end{itemize}

\subsubsection{Preliminary design}

\begin{itemize}
    \item Build up laminate using behaviours of individual layers
    \item Utilise carpet plots where possible
    \item Calculate and update thickness of laminate based on constraints
\end{itemize}

\subsubsection{Detailed design}

\begin{itemize}
    \item Use laminate theory to check mixed failure criteria and various loading patterns, using material properties collected
    \item Optimise where necessary and/or possible
    \item Implement other design considerations such as impact resistance.
\end{itemize}

\subsubsection{Development of Objectives}

\begin{itemize}
    \item Analyse and process designs to display in an informative and visual manner
    \item Formulate design suggestions with design trade-offs for the different flax-carbon arrangements
    \item Extended analysis if feasible
\end{itemize}


\section{Summary}

This project aims to explore the viability of flax-carbon hybrid composites for use in bicycle components through computer modelling and analysis. By collating material data from the literature and implementing them in modelling tools such as CCSM, the model will evaluate how different layup configurations affect mechanical and vibrational performance under simplified loading scenarios. 


\bibliographystyle{plainnat}
\begin{thebibliography}{9}

\bibitem{Bcomp} Bcomp \url{https://www.bcomp.com/applications/sports/}.



\end{thebibliography}

\begin{figure*}[h!]
    \centering
    \includegraphics[width=0.9\textwidth]{Gantt.png} 
    \caption{Gantt Chart for project timeline}
    \label{fig:Gantt}
\end{figure*}

\clearpage

\includepdf[pages=-]{SSD.pdf}

\end{document}
