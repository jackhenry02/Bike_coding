\documentclass[11pt,twocolumn]{article}
\usepackage[utf8]{inputenc}
\usepackage{graphicx}
\usepackage{amsmath}
\usepackage{hyperref}
\usepackage{natbib}
\usepackage{geometry}
\geometry{margin=1in}
\usepackage{subcaption}
\usepackage{tikz}
\usetikzlibrary{3d, arrows.meta}
\usepackage[none]{hyphenat}

\title{Literature Review: Computer Modelling of Flax-Carbon Hybrid Composites for Bicycle Applications}
\author{Jack Henry}
\date{\today}

\begin{document}
\maketitle

\section{Introduction}
The bicycle industry has seen significant evolution in materials technology, particularly with the adoption of composite materials. This review examines the current state of computer modelling and simulation techniques for flax-carbon hybrid composites, with a focus on bicycle applications. The integration of natural fibres, particularly flax, with carbon fibre represents a promising direction for sustainable yet high-performance bicycle components.

\section{Evolution of Composite Materials in Bicycle Design}
\subsection{Historical Development}
The use of composite materials in bicycles has evolved significantly since their introduction in the 1970s. Initially dominated by carbon fibre, the industry has seen a gradual shift towards more sustainable alternatives. This section examines the historical progression of composite materials in bicycle design, from early carbon fibre applications to current hybrid approaches.

\subsection{Current State of Composite Bicycle Components}
Modern bicycle design heavily relies on composite materials, particularly for high-performance components. This section reviews the current applications of composites in bicycle design, focusing on structural components like forks, frames, and handlebars. Special attention is given to the performance requirements and design considerations specific to bicycle applications.

\section{Natural Fibre Composites: Focus on Flax}
\subsection{Properties and Characteristics of Flax Fibre}
Flax fibre composites offer unique advantages in terms of sustainability and mechanical properties. This section examines the fundamental properties of flax fibre, including its mechanical characteristics, damping properties, and environmental benefits. The discussion includes comparative analysis with traditional carbon fibre properties.

\subsection{Challenges and Limitations}
Despite their advantages, flax fibre composites face several challenges in high-performance applications. This section discusses the limitations of flax fibre composites, including moisture sensitivity, variability in properties, and processing challenges. The discussion includes current research efforts to address these limitations.

\section{Hybrid Composites: Flax-Carbon Combinations}
\subsection{Mechanical Performance}
The combination of flax and carbon fibres offers a unique opportunity to balance performance and sustainability. This section reviews studies on the mechanical properties of flax-carbon hybrid composites, including strength, stiffness, and fatigue characteristics. Special attention is given to the effects of different fibre ratios and layup configurations.

\subsection{Environmental and Economic Considerations}
The environmental benefits of incorporating flax fibres into composite structures are significant. This section examines the life cycle analysis of flax-carbon hybrid composites, including manufacturing processes, end-of-life considerations, and economic feasibility.

\section{Computer Modelling and Simulation Techniques}
\subsection{Current Modelling Approaches}
Computer simulation plays a crucial role in the development and optimization of composite materials. This section reviews current modelling techniques for composite materials, including finite element analysis, classical laminate theory, and multi-scale modelling approaches. The discussion includes the specific challenges of modelling hybrid composites.

\subsection{Software Tools and Platforms}
Various software tools are available for composite material simulation. This section examines the capabilities and limitations of current software platforms, including specialized composite analysis tools and general-purpose finite element software. The discussion includes considerations for modelling hybrid composites.

\section{Application to Bicycle Components}
\subsection{Structural Analysis Requirements}
Bicycle components, particularly the fork, present unique challenges for composite design. This section reviews the specific requirements for bicycle component simulation, including load cases, failure criteria, and performance metrics. The discussion includes considerations for both static and dynamic analysis.

\subsection{Case Studies and Applications}
Several studies have examined the application of hybrid composites in bicycle components. This section reviews relevant case studies, focusing on the simulation and testing of flax-carbon hybrid components. The discussion includes both successful implementations and lessons learned.

\section{Current Challenges and Future Directions}
\subsection{Modelling Challenges}
The simulation of hybrid composites presents several technical challenges. This section discusses current limitations in modelling techniques, including the representation of fibre-matrix interactions, damage progression, and environmental effects.

\subsection{Research Gaps and Opportunities}
Despite significant progress, several areas require further research. This section identifies key research gaps in the field, including the need for improved modelling techniques, better understanding of hybrid composite behavior, and development of standardized testing methods.

\section{Conclusion}
The integration of flax and carbon fibres in bicycle components represents a promising direction for sustainable yet high-performance design. While significant progress has been made in both material development and computer modelling, several challenges remain. Future research should focus on improving modelling techniques, understanding long-term performance, and developing standardized approaches to hybrid composite design.

\bibliographystyle{plainnat}
\bibliography{references}

\end{document} 